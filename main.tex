\documentclass{article}
\usepackage[utf8]{inputenc}
\usepackage[spanish]{babel}
\usepackage{listings}
\usepackage{graphicx}
\graphicspath{ {images/} }
\usepackage{cite}

\begin{document}

\begin{titlepage}
    \begin{center}
        \vspace*{1cm}
            
        \Huge
        \textbf{Calistenia}
            
        \vspace{0.5cm}
        \LARGE
            
        \vspace{1.5cm}
            
        \textbf{Juan Pablo Rendón Jiménez}
            
        \vfill
            
        \vspace{0.8cm}
            
        \Large
        Despartamento de Ingeniería Electrónica y Telecomunicaciones\\
        Universidad de Antioquia\\
        Medellín\\
        Marzo de 2021
            
    \end{center}
\end{titlepage}

\tableofcontents
\newpage
        
\section{Primer paso}\label{intro}
Con una sola mano, levantar la hoja, coger las dos tarjetas y poner las tarjetas debajo de la hoja.

\section{Segundo paso} \label{contenido}
Utilizando la misma mano, pegar las tarjetas y sostenerlas, con el dedo pulgar e indice, del lado mas pequeño. Luego, con el anular separas las tarjetas de modo que en la base haya una separacion de 5 centimetros, pero teniendo en cuenta que la parte superior de las tarjetas tienen que estar pegadas. 

\section{Tercer paso} \label{imagenes}
El desafio termina cuando las tarjetas queden equilibradas y no se caigan.


\end{document}
